\documentclass{article}

\usepackage{xeCJK}
\setCJKmainfont{SimSun}
\usepackage{xcolor}
\usepackage{hyperref}

\title{Network Multimedia Homework V\thanks{%
        This homework report is source-opened on
        \href{https://github.com/Qinka/multimedia-homework}{GitHub:Qinka/multimedia-homework}.
        Because one of my friend's Java homework, which is also opened on GitHub, was plagiarized by others, I just want declare the my, 李约瀚, copyleft(right) of the report and my homework.
        I, John Lee, is the only writer of this report.
    }}
\author{李约瀚 \\ 14130140331 \\ qinka@live.com \\ me@qinka.pro}

%% the definitions of the definition
%%%  Books
\def\vFundamentalsOfMutimedia{\textit{Fundamentals of Multimedia}}
\def\vConcreteMathematics{\textit{Concrete Mathematics}}

\begin{document}
    
    \maketitle
    
    \section{No.5, Chapter 8}
    According to \vFundamentalsOfMutimedia, the Eq.(8.15) shows the transforming of the \verb|DCT|.
    For the 2-D \verb|DCT|, the equlation is 
    \begin{equation}
    \label{eq:dct:baseeq}
    F(u,v) = \frac{2C(u)C(v)}{\sqrt{MN}}\sum\limits_{i=0}^{M-1}\sum\limits_{j=0}^{N-1}\cos\frac{(2i+1)u\pi}{2M}\cos\frac{(2j+1)v\pi}{2N}f(i,j)
    \end{equation}
    
    where the $i,u = 0,1,\dots,M-1$, $j,v = 0,1,\dots,N-1$, and the constant C(u) and C(v) are defined as:
    \begin{equation}
    C(\xi) = \left\{
    \begin{array}{cl}
    \frac{\sqrt{2}}{2} & \xi = 0, \\
    1 & otherwise.
    \end{array}
    \right.
    \end{equation}
    
    \subsection{Question (a)}
    
    The largest coefficient of \verb|DCT| is the \verb|DC| value, which is $8 \times 255 = 2040$.
    The value $255$ means that each value of the image is the $255$, which is usually
    a white image. And the other \verb|AC| values are the zeros.
    
    \subsection{Question (b)}
    
    Because of the value of every pixel of the image is same, the \verb|AC| values in the \verb|DCT| are zeros. 
    When the value of the pixel is change, they are still same. So there is no effect on the $F[2.3]$, and so other \verb|AC| values are.
    
    \subsection{Question (c)}
    
    Such operation do reduce bits. For example, if there is an image, whose range of
    the value is 100 ~ 156, when we encode such image we might need a lots of bits,
    might say 8 bits. However when we transform the image to a \textbf{zero mean image}
    we will still need some bits but less, might say 6 bits.
    
    \subsection{Question (d)}
    
    According to the decoding, the best but simple way is adding the $128$ back.
    
    \section{No.2, Chapter 9}
    
    So the get the equation:
    \begin{equation}
    \label{eq:dctn:base}
        F_N(u,v) =
            \frac{2C(u)C(v)}{N}
            \sum\limits_{i=0}^{N-1}\sum\limits_{j=0}^{N-1}
            \cos\frac{(2i+1)u\pi}{2N}\cos\frac{(2j+1)v\pi}{2N}
            f(i,j)
    \end{equation} 
    where the $i,u = 0,1,\dots,N-1$, $j,v = 0,1,\dots,N-1$, and the constant C(u) and C(v) are defined as:
    \begin{equation}
        C(\xi) = \left\{
        \begin{array}{cl}
        \frac{\sqrt{2}}{2} & \xi = 0, \\
        1 & otherwise.
        \end{array}
        \right.
    \end{equation}
    
    According to \vConcreteMathematics's Eq.(2.28), the Eq.(\ref{eq:dctn:base}) will
    be transform to
    \begin{equation}
        \label{eq:dctn:base2}
        F_N(u,v)=%
            \frac{2C(u)C(v)}{N}%
            \left(\sum\limits_{p=0}^{N-1}\cos\frac{(2p+1)u\pi}{2N}\right)^2%
            f(i,j)
    \end{equation}
    
    For $\widetilde{F_2}$, the equation will be
    $$
        \widetilde{F_2}(u,v) =
            C(u)C(v)
            \left(\sum\limits_{p=0}^{1}\cos\frac{(2p+1)u\pi}{4}\right)^2%
    $$
    That means
    \begin{equation}
        \label{eq:dctn:tt}
        \widetilde{F_2}(u,v) = 
            C(u)C(v)
            \left(
                \cos\frac{u\pi}{4}^2f(0,0)
              + 2\cos\frac{u\pi}{4}\cos\frac{3u\pi}{4}
              + \cos\frac{3u\pi}{4}^2f(1,1)
            \right)
    \end{equation}
    
    So we can evaluate the $F_2$, according to Eq.(\ref{eq:dctn:tt})
    \begin{equation}
    \label{eq:dctn:f2}
    F_2(u,v) = \left\{
        \begin{array}{rl}
        0 & u=0,\,v=0, \\
        0 & u=1,\,v=0, \\
        200 & u=0,\,v=1, \\
        0 & u=1,\,v=1. 
        \end{array}
    \right.
    \end{equation}
    
    According to Eq.(\ref{eq:dctn:tt}) we can get the result
    $$
        F_2(u,v):\, \left\{ \begin{array}{rrrrrrrr}
        0 & 200 & 0 & 200 & 0 & 200 & 0 & 200 \\
        0 & 0 & 0 & 0 & 0 & 0 & 0 & 0 \\
        0 & 200 & 0 & 200 & 0 & 200 & 0 & 200 \\
        0 & 0 & 0 & 0 & 0 & 0 & 0 & 0 \\
        0 & 200 & 0 & 200 & 0 & 200 & 0 & 200 \\
        0 & 0 & 0 & 0 & 0 & 0 & 0 & 0 \\
        0 & 200 & 0 & 200 & 0 & 200 & 0 & 200 \\
        0 & 0 & 0 & 0 & 0 & 0 & 0 & 0         
        \end{array} \right\}
    $$
    
    
    
    
    
    
    
\end{document}