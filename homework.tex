\documentclass{ctexart}

\title{网络多媒体作业 1}
\author{李约瀚 \\ 14130140331 \\ qinka@qinka.pw \\ qinka@live.com}

\begin{document}
    \maketitle
    \paragraph{弹幕}
    弹幕的原本定义是一种炮兵战术:一种炮兵基本战术,简单地说,就是步兵在冲锋或前进的时候,炮兵按一定顺序延伸炮火,始终把炮弹打到步兵前面数百米的地方,为步兵提供火力掩护的战术。在互联网中多指一种观影评论的方式。这个观影评论会“实时\footnote{相对于视屏的时间轴}”显示在其他观影者的画面上。
    弹幕对于有画面的视频,某些时候是一种干扰。但是对于设计良好的弹幕系统,能较为良好的避免这样的问题。弹幕能弥补通常单一的视频播放方式的诸多缺点。
    对于没有字幕的视频,弹幕的出现能较好的解决字幕缺失的情况。而对于一些视频中出现一些非生活常识的内容,
    观影者的自愿行为可以帮助其他观影者了解知识背景。同时互联网中,无组织的这样一群弹幕发送者,很多时候能表达真实的观众对视频的体验。
    而当视频中出现了一些观影者普遍愿意跳过的内容,比如说视频中最无聊的一部分,在弹幕中能找出来适当的继续观看的时间节点。
    对于纯音乐视频,弹幕除了充当上述功能之外,对于鬼畜,一般还有计数功能。弹幕作为一种新的多媒体方式,或者说是伴随产物,成为了当今互联网多媒体中的
    比不可少的组成内容。
    \paragraph{鬼畜}
    鬼畜是一种“原创”视频类型。百度百科中是这样描述的:“该类视频以高度同步、快速重复的素材配合BGM的节奏鬼一样的抽搐来达到洗脑或喜感效果,或通过视频(或音频)剪辑,用频率极高的重复画面(或声音)组合而成的一段节奏配合音画同步率极高的一类视频。”
    鬼畜来源于日本弹幕视频网站NICONICO动画,其原本的名字为应为音MAD。而音MAD,是一种使用素材中的乐器对所选背景音乐进行演奏的视频形式。
    在鬼畜的视频中,通常则是使用一些“经典”的视频,将其声音作为素材,加上重现选择的背景音乐,组合而成。其中还有使用“谐音”的方式制作该类视频。
    例如某国外拍摄的有关于苏联最后攻克纳粹德国首都柏林的影片中,希特勒在地堡中所说的德文中有部分与中文谐音,其中包括类似中文方言的“气死我了”
    这样的声音许多鬼畜制作者将之用到制作的鬼畜中。其中还有一些是视频作者使用经典的视频片段中出现的声音,将之音调变换契合新的背景音乐制作成视频。
    例如将经典的《猫和老鼠》中一些击打或者口哨之类的声音,将音调调整适应《命运交响曲》,制作出来的视频。而传统“音MAD”除了使用乐器之外,还衍生出使用
    包括游戏中的枪炮声的通过组合制作出来的音乐/视频。
    \paragraph{bilibili}
    bilibili 是国内最大的基于年轻人\footnote{事实上并不能这样定义,因为诸如张绍忠这样的人物也在其中。}文化视频网站与直播平台,又称“B站”。
    B站 的特色在于其弹幕引擎\footnote{开放源码的,在 GitHub 上}。同时与其他国内外的视频网站不同的是: B站的视频,无论是访客还是用户,都是没有广告的。
    相对的,B站设置有诸多分区,比如番剧、科技、生活、鬼畜、广告等。将广告作为作为一种艺术与其他内容一样呈现给大家。而 B站一方面是从“宅”这种文化发展来的,
    但是却是朝向积极方面发展。
    
\end{document}