\documentclass{ctexart}

\author{李约瀚 \\ 14130140331 \\ qinka@live.com \\ qinka@qinka.pw}

\title{Multimedia Homework 3}

\usepackage{listings}

\begin{document}
\maketitle

\section{Chapter 3}

Usually each pixel can be encoded with a 24-bit data, and each of RED, BLUE, and GREEN has 8 bits. But people are more sensitive to red and green than blue.
So we can allocate more bits to red and green. In fact, if we define $s$ as the level of sensitivity. There is a conclusion:
$$s_{red} = s_{green} = 1.5s_{blue}$$
Red and green will get 3 point for weigth for each, and blue will get 2. So each point of weight will get 3 bits.
Then red should have 9 bits, so green does, and blue should have 6 bits.

\section{Chapter 4}

According to (4.6), it from $\overline{x}(\lambda)$, $\overline{y}(\lambda)$, and $\overline{z}(\lambda)$.
And the code for count is here:
\begin{lstlisting}[language=Matlab]
function colors (x,y,z,be,ed)
  waves = (be:ed)';
  figure;
  plot(waves,x,waves,y,waves,z);
  figure;
  plot3(x,y,z,'*');
  d = x + y + z;
  xy = zeros(ed-bg+1,3);
  xyz = [x,y,z];
  for i = 1:3
    xy(:,i) = xyz(:,k) ./ d;
  end
  figure;
  plot3(xy(:,1),xy(:,2),xy(:,3),'*');
end
\end{lstlisting}

\end{document}